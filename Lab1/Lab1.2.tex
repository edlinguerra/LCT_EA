% Options for packages loaded elsewhere
\PassOptionsToPackage{unicode}{hyperref}
\PassOptionsToPackage{hyphens}{url}
\PassOptionsToPackage{dvipsnames,svgnames,x11names}{xcolor}
%
\documentclass[
  letterpaper,
  DIV=11,
  numbers=noendperiod]{scrartcl}

\usepackage{amsmath,amssymb}
\usepackage{lmodern}
\usepackage{iftex}
\ifPDFTeX
  \usepackage[T1]{fontenc}
  \usepackage[utf8]{inputenc}
  \usepackage{textcomp} % provide euro and other symbols
\else % if luatex or xetex
  \usepackage{unicode-math}
  \defaultfontfeatures{Scale=MatchLowercase}
  \defaultfontfeatures[\rmfamily]{Ligatures=TeX,Scale=1}
\fi
% Use upquote if available, for straight quotes in verbatim environments
\IfFileExists{upquote.sty}{\usepackage{upquote}}{}
\IfFileExists{microtype.sty}{% use microtype if available
  \usepackage[]{microtype}
  \UseMicrotypeSet[protrusion]{basicmath} % disable protrusion for tt fonts
}{}
\makeatletter
\@ifundefined{KOMAClassName}{% if non-KOMA class
  \IfFileExists{parskip.sty}{%
    \usepackage{parskip}
  }{% else
    \setlength{\parindent}{0pt}
    \setlength{\parskip}{6pt plus 2pt minus 1pt}}
}{% if KOMA class
  \KOMAoptions{parskip=half}}
\makeatother
\usepackage{xcolor}
\setlength{\emergencystretch}{3em} % prevent overfull lines
\setcounter{secnumdepth}{-\maxdimen} % remove section numbering
% Make \paragraph and \subparagraph free-standing
\ifx\paragraph\undefined\else
  \let\oldparagraph\paragraph
  \renewcommand{\paragraph}[1]{\oldparagraph{#1}\mbox{}}
\fi
\ifx\subparagraph\undefined\else
  \let\oldsubparagraph\subparagraph
  \renewcommand{\subparagraph}[1]{\oldsubparagraph{#1}\mbox{}}
\fi

\usepackage{color}
\usepackage{fancyvrb}
\newcommand{\VerbBar}{|}
\newcommand{\VERB}{\Verb[commandchars=\\\{\}]}
\DefineVerbatimEnvironment{Highlighting}{Verbatim}{commandchars=\\\{\}}
% Add ',fontsize=\small' for more characters per line
\usepackage{framed}
\definecolor{shadecolor}{RGB}{241,243,245}
\newenvironment{Shaded}{\begin{snugshade}}{\end{snugshade}}
\newcommand{\AlertTok}[1]{\textcolor[rgb]{0.68,0.00,0.00}{#1}}
\newcommand{\AnnotationTok}[1]{\textcolor[rgb]{0.37,0.37,0.37}{#1}}
\newcommand{\AttributeTok}[1]{\textcolor[rgb]{0.40,0.45,0.13}{#1}}
\newcommand{\BaseNTok}[1]{\textcolor[rgb]{0.68,0.00,0.00}{#1}}
\newcommand{\BuiltInTok}[1]{\textcolor[rgb]{0.00,0.23,0.31}{#1}}
\newcommand{\CharTok}[1]{\textcolor[rgb]{0.13,0.47,0.30}{#1}}
\newcommand{\CommentTok}[1]{\textcolor[rgb]{0.37,0.37,0.37}{#1}}
\newcommand{\CommentVarTok}[1]{\textcolor[rgb]{0.37,0.37,0.37}{\textit{#1}}}
\newcommand{\ConstantTok}[1]{\textcolor[rgb]{0.56,0.35,0.01}{#1}}
\newcommand{\ControlFlowTok}[1]{\textcolor[rgb]{0.00,0.23,0.31}{#1}}
\newcommand{\DataTypeTok}[1]{\textcolor[rgb]{0.68,0.00,0.00}{#1}}
\newcommand{\DecValTok}[1]{\textcolor[rgb]{0.68,0.00,0.00}{#1}}
\newcommand{\DocumentationTok}[1]{\textcolor[rgb]{0.37,0.37,0.37}{\textit{#1}}}
\newcommand{\ErrorTok}[1]{\textcolor[rgb]{0.68,0.00,0.00}{#1}}
\newcommand{\ExtensionTok}[1]{\textcolor[rgb]{0.00,0.23,0.31}{#1}}
\newcommand{\FloatTok}[1]{\textcolor[rgb]{0.68,0.00,0.00}{#1}}
\newcommand{\FunctionTok}[1]{\textcolor[rgb]{0.28,0.35,0.67}{#1}}
\newcommand{\ImportTok}[1]{\textcolor[rgb]{0.00,0.46,0.62}{#1}}
\newcommand{\InformationTok}[1]{\textcolor[rgb]{0.37,0.37,0.37}{#1}}
\newcommand{\KeywordTok}[1]{\textcolor[rgb]{0.00,0.23,0.31}{#1}}
\newcommand{\NormalTok}[1]{\textcolor[rgb]{0.00,0.23,0.31}{#1}}
\newcommand{\OperatorTok}[1]{\textcolor[rgb]{0.37,0.37,0.37}{#1}}
\newcommand{\OtherTok}[1]{\textcolor[rgb]{0.00,0.23,0.31}{#1}}
\newcommand{\PreprocessorTok}[1]{\textcolor[rgb]{0.68,0.00,0.00}{#1}}
\newcommand{\RegionMarkerTok}[1]{\textcolor[rgb]{0.00,0.23,0.31}{#1}}
\newcommand{\SpecialCharTok}[1]{\textcolor[rgb]{0.37,0.37,0.37}{#1}}
\newcommand{\SpecialStringTok}[1]{\textcolor[rgb]{0.13,0.47,0.30}{#1}}
\newcommand{\StringTok}[1]{\textcolor[rgb]{0.13,0.47,0.30}{#1}}
\newcommand{\VariableTok}[1]{\textcolor[rgb]{0.07,0.07,0.07}{#1}}
\newcommand{\VerbatimStringTok}[1]{\textcolor[rgb]{0.13,0.47,0.30}{#1}}
\newcommand{\WarningTok}[1]{\textcolor[rgb]{0.37,0.37,0.37}{\textit{#1}}}

\providecommand{\tightlist}{%
  \setlength{\itemsep}{0pt}\setlength{\parskip}{0pt}}\usepackage{longtable,booktabs,array}
\usepackage{calc} % for calculating minipage widths
% Correct order of tables after \paragraph or \subparagraph
\usepackage{etoolbox}
\makeatletter
\patchcmd\longtable{\par}{\if@noskipsec\mbox{}\fi\par}{}{}
\makeatother
% Allow footnotes in longtable head/foot
\IfFileExists{footnotehyper.sty}{\usepackage{footnotehyper}}{\usepackage{footnote}}
\makesavenoteenv{longtable}
\usepackage{graphicx}
\makeatletter
\def\maxwidth{\ifdim\Gin@nat@width>\linewidth\linewidth\else\Gin@nat@width\fi}
\def\maxheight{\ifdim\Gin@nat@height>\textheight\textheight\else\Gin@nat@height\fi}
\makeatother
% Scale images if necessary, so that they will not overflow the page
% margins by default, and it is still possible to overwrite the defaults
% using explicit options in \includegraphics[width, height, ...]{}
\setkeys{Gin}{width=\maxwidth,height=\maxheight,keepaspectratio}
% Set default figure placement to htbp
\makeatletter
\def\fps@figure{htbp}
\makeatother

\KOMAoption{captions}{tableheading}
\makeatletter
\makeatother
\makeatletter
\makeatother
\makeatletter
\@ifpackageloaded{caption}{}{\usepackage{caption}}
\AtBeginDocument{%
\ifdefined\contentsname
  \renewcommand*\contentsname{Table of contents}
\else
  \newcommand\contentsname{Table of contents}
\fi
\ifdefined\listfigurename
  \renewcommand*\listfigurename{List of Figures}
\else
  \newcommand\listfigurename{List of Figures}
\fi
\ifdefined\listtablename
  \renewcommand*\listtablename{List of Tables}
\else
  \newcommand\listtablename{List of Tables}
\fi
\ifdefined\figurename
  \renewcommand*\figurename{Figure}
\else
  \newcommand\figurename{Figure}
\fi
\ifdefined\tablename
  \renewcommand*\tablename{Table}
\else
  \newcommand\tablename{Table}
\fi
}
\@ifpackageloaded{float}{}{\usepackage{float}}
\floatstyle{ruled}
\@ifundefined{c@chapter}{\newfloat{codelisting}{h}{lop}}{\newfloat{codelisting}{h}{lop}[chapter]}
\floatname{codelisting}{Listing}
\newcommand*\listoflistings{\listof{codelisting}{List of Listings}}
\makeatother
\makeatletter
\@ifpackageloaded{caption}{}{\usepackage{caption}}
\@ifpackageloaded{subcaption}{}{\usepackage{subcaption}}
\makeatother
\makeatletter
\@ifpackageloaded{tcolorbox}{}{\usepackage[many]{tcolorbox}}
\makeatother
\makeatletter
\@ifundefined{shadecolor}{\definecolor{shadecolor}{rgb}{.97, .97, .97}}
\makeatother
\makeatletter
\makeatother
\ifLuaTeX
  \usepackage{selnolig}  % disable illegal ligatures
\fi
\IfFileExists{bookmark.sty}{\usepackage{bookmark}}{\usepackage{hyperref}}
\IfFileExists{xurl.sty}{\usepackage{xurl}}{} % add URL line breaks if available
\urlstyle{same} % disable monospaced font for URLs
\hypersetup{
  pdftitle={Estadística descriptiva con R},
  pdfauthor={Prof.~Edlin Guerra Castro},
  colorlinks=true,
  linkcolor={blue},
  filecolor={Maroon},
  citecolor={Blue},
  urlcolor={Blue},
  pdfcreator={LaTeX via pandoc}}

\title{Estadística descriptiva con R}
\author{Prof.~Edlin Guerra Castro}
\date{30/08/2022}

\begin{document}
\maketitle
\ifdefined\Shaded\renewenvironment{Shaded}{\begin{tcolorbox}[enhanced, frame hidden, borderline west={3pt}{0pt}{shadecolor}, interior hidden, breakable, sharp corners, boxrule=0pt]}{\end{tcolorbox}}\fi

\hypertarget{explorando-datos-r}{%
\subsection{Explorando datos R}\label{explorando-datos-r}}

El objetivo principal de este laboratorio es introducirlos a la
exploración de datos y estadística descriptiva con
\href{https://www.r-project.org/}{R} y
\href{https://rstudio.com/}{RStudio}, las herramientas computacionales
que utilizaremos a lo largo del semestre para aprender a aplicar los
conceptos más importantes de \emph{Estadística Aplicada}, pero en
especial, para aprender a procesar y analizar datos reales.

Los paquetes que necesitaremos para esta actividad son \texttt{dplyr},
\texttt{ggplot2} y \texttt{moments}. En caso no tenga esto paquetes, los
puede instalar con los siguientes códigos:

\begin{Shaded}
\begin{Highlighting}[]
\CommentTok{\#Para instalar ggplot2 (realizar gráficos de alta calidad)}
\FunctionTok{install.packages}\NormalTok{(}\StringTok{"ggplot2"}\NormalTok{)}

\CommentTok{\#Para administrar bases de datos: usa dplyr}
\FunctionTok{install.packages}\NormalTok{(}\StringTok{"dplyr"}\NormalTok{)}

\CommentTok{\#para estimar simetría y curtosis}
\FunctionTok{install.packages}\NormalTok{(}\StringTok{"moments"}\NormalTok{)}
\end{Highlighting}
\end{Shaded}

Alternativamente, puedes usar la ventala \emph{Tools/install
packages\ldots{}}, se desplegará una ventana para que escribas el nombre
del paquete a instalar. Los paquetes se instalan una sola vez, siempre
que estes en el mismo computador. Para usarlos debes incluirlos en tu
sesión de trabajo cada vez que se inicia la sesión. Esto se logra con la
función \texttt{library}:

\begin{Shaded}
\begin{Highlighting}[]
\FunctionTok{library}\NormalTok{(}\StringTok{"ggplot2"}\NormalTok{) }\CommentTok{\# repita este código reemplazando para cada paquete}
\end{Highlighting}
\end{Shaded}

Hagamos un análisis exploratorio a los datos de los
\href{https://doi.org/10.1371/journal.pone.0090081}{pinguinos de la
Antártida del género \emph{Pygoscelis}}. Lo primero que debe hacer es
instalar el paquete de datos \texttt{palmerpenguins} y cárguelo en su
sesión. Luego haga lo siguiente:

\begin{enumerate}
\def\labelenumi{\arabic{enumi}.}
\tightlist
\item
  Busque en la pestaña \emph{Help} qué es \emph{palmerpenguin}.
\item
  Identifique la base de datos \texttt{penguins} y cárguela en su
  ambiente global con \texttt{data(penguins)}.
\item
  Identifique cuántas variables hay, cuál es la naturaleza de cada una
  de ellas (tipo de variable, escala), así como cuáles pueden ser
  consideradas causales y cuáles respuesta.
\item
  Efectue un gráfico de dispersión entre las variables
  \texttt{body\_mass\_g} y \texttt{flipper\_lemgth\_mm}. Mida el grado
  de asociación ¿cómo lo haría? Una forma de hacer estas cosas es
  graficando la asociación, y estimando la correlación (método que se
  desarrollará en otro laboratorio, pero acá es pide con fines
  demostrativos)
\end{enumerate}

\begin{Shaded}
\begin{Highlighting}[]
\FunctionTok{plot}\NormalTok{(penguins}\SpecialCharTok{$}\NormalTok{body\_mass\_g, penguins}\SpecialCharTok{$}\NormalTok{flipper\_length\_mm)}

\CommentTok{\#¿qué hace cor()?}
\FunctionTok{cor}\NormalTok{(penguins}\SpecialCharTok{$}\NormalTok{body\_mass\_g, penguins}\SpecialCharTok{$}\NormalTok{flipper\_length\_mm)}

\CommentTok{\#Si no obtuvo resultado, trate de resulverlo con el argumento "use"}
\end{Highlighting}
\end{Shaded}

Usemos el paquete \texttt{ggplot2} para mejorar el gráfico:

\begin{Shaded}
\begin{Highlighting}[]
\NormalTok{pp }\OtherTok{\textless{}{-}} \FunctionTok{ggplot}\NormalTok{(}\AttributeTok{data =}\NormalTok{ penguins, }\FunctionTok{aes}\NormalTok{(}\AttributeTok{x =}\NormalTok{ flipper\_length\_mm,}
                                  \AttributeTok{y =}\NormalTok{ body\_mass\_g, }
                                  \AttributeTok{colour =}\NormalTok{ species))}\SpecialCharTok{+}
      \FunctionTok{geom\_point}\NormalTok{()}
  
\NormalTok{pp  }
  
\CommentTok{\#Mejoremos con capas}
\NormalTok{pp }\SpecialCharTok{+}  \FunctionTok{theme\_bw}\NormalTok{()}\SpecialCharTok{+}
      \FunctionTok{xlab}\NormalTok{(}\StringTok{"Largo de aleta (mm)"}\NormalTok{)}\SpecialCharTok{+}
      \FunctionTok{ylab}\NormalTok{(}\StringTok{"Masa corporal (g)"}\NormalTok{)}\SpecialCharTok{+}
      \FunctionTok{scale\_y\_continuous}\NormalTok{(}\AttributeTok{breaks =} \FunctionTok{seq}\NormalTok{(}\DecValTok{2600}\NormalTok{,}\DecValTok{6400}\NormalTok{,}\DecValTok{400}\NormalTok{))}\SpecialCharTok{+}
      \FunctionTok{scale\_x\_continuous}\NormalTok{(}\AttributeTok{breaks =} \FunctionTok{seq}\NormalTok{(}\DecValTok{170}\NormalTok{,}\DecValTok{240}\NormalTok{,}\DecValTok{5}\NormalTok{))}
      
\CommentTok{\#¿cuál gráfico le gustó más? }
\end{Highlighting}
\end{Shaded}

Llegados a este punto, vamos hacer algo de estadística descriptiva. Esta
es la parte que deberán entregar como tarea. Calculen promedio,
varianza, desviación estandar, valor mínimo y máximo, cuartiles,
simetría y curtosis a la variable masa corporal. Usen para ello las
funciones recomendadas y responda las siguientes preguntas:

\begin{Shaded}
\begin{Highlighting}[]
\CommentTok{\# Para facilitar cálculos, vamos a remover los datos sin registro (identificados con NA), usando el siguiente código:}

\NormalTok{penguins2 }\OtherTok{\textless{}{-}}\NormalTok{ penguins }\SpecialCharTok{\%\textgreater{}\%}
  \FunctionTok{na.omit}\NormalTok{()}
\end{Highlighting}
\end{Shaded}

\hypertarget{preguntas}{%
\subsubsection{PREGUNTAS}\label{preguntas}}

\begin{enumerate}
\def\labelenumi{\arabic{enumi}.}
\tightlist
\item
  Copia el comando que sigue. ¿Qué se calculó?
\end{enumerate}

\begin{Shaded}
\begin{Highlighting}[]
\NormalTok{xx}\OtherTok{\textless{}{-}}\NormalTok{penguins2}\SpecialCharTok{$}\NormalTok{body\_mass\_g}
\FunctionTok{sum}\NormalTok{(xx, }\AttributeTok{na.rm =}\NormalTok{ T)}\SpecialCharTok{/}\FunctionTok{length}\NormalTok{(xx)}
\end{Highlighting}
\end{Shaded}

\begin{enumerate}
\def\labelenumi{\arabic{enumi}.}
\setcounter{enumi}{1}
\item
  Busca y aplica una función que ejecute la linea de comando anterior.
  PISTA: escribe \texttt{?mean} en la cónsola y enter para ampliar tu
  búsqueda.
\item
  Calcula la mediana de la masa corporal usando la función
  correspondiente.
\item
  Calcula la varianza y desviación estándar de la masa corporal usando
  el comando \texttt{var()}
\item
  ¿Cuál es la diferencia entre estas dos fórmulas? ¿Representan lo
  mismo?
\end{enumerate}

\begin{Shaded}
\begin{Highlighting}[]
\FunctionTok{sum}\NormalTok{((xx}\SpecialCharTok{{-}}\FunctionTok{mean}\NormalTok{(xx))\textbackslash{}}\SpecialCharTok{\^{}}\DecValTok{2}\NormalTok{)}\SpecialCharTok{/}\FunctionTok{length}\NormalTok{(xx)}

\FunctionTok{sum}\NormalTok{((xx}\SpecialCharTok{{-}}\FunctionTok{mean}\NormalTok{(xx))\textbackslash{}}\SpecialCharTok{\^{}}\DecValTok{2}\NormalTok{)}\SpecialCharTok{/}\NormalTok{(}\FunctionTok{length}\NormalTok{(xx)}\SpecialCharTok{{-}}\DecValTok{1}\NormalTok{)}
\end{Highlighting}
\end{Shaded}

\begin{enumerate}
\def\labelenumi{\arabic{enumi}.}
\setcounter{enumi}{5}
\item
  Con base en el valor de la varianza y usando operadores aritméticos,
  calcula la desviación estándar de la masa corporal. Confirma tu
  resultado usando la función \texttt{sd()}.
\item
  Explore el rango de la masa corporal identificando mínimos y máximos
  con la función \texttt{min()} y \texttt{max()}, respectivamente.
\item
  Ahora estime los cuartiles de la masa corporal con la función
  \texttt{quantile()}
\item
  Describa la forma de la distribución de la masa corporal usando la
  simetría y curtosis con las funciones \texttt{skewness()} y
  \texttt{kurtosis()}.
\item
  Todas estas estimaciones ignoran las posibles diferencias en la masa
  corporal entre las especies. ¿Qué le dice este gráfico?
\end{enumerate}

\begin{Shaded}
\begin{Highlighting}[]
\FunctionTok{boxplot}\NormalTok{(body\_mass\_g}\SpecialCharTok{\textasciitilde{}}\NormalTok{species, }\AttributeTok{data =}\NormalTok{ penguins2)}
\end{Highlighting}
\end{Shaded}

\begin{enumerate}
\def\labelenumi{\arabic{enumi}.}
\setcounter{enumi}{10}
\tightlist
\item
  Calcule estos estimadores para cada especie usando el paquete
  \textbf{dplyr} y sus funciones \texttt{group\_by()} y
  \texttt{summarize()}. Estas líneas de comando lo ayudarán (interprete
  los resultados):
\end{enumerate}

\begin{Shaded}
\begin{Highlighting}[]
\FunctionTok{library}\NormalTok{(dplyr)}

\NormalTok{penguins }\SpecialCharTok{\%\textgreater{}\%} 
  \FunctionTok{group\_by}\NormalTok{(species) }\SpecialCharTok{\%\textgreater{}\%} 
  \FunctionTok{summarise}\NormalTok{(}\AttributeTok{media =} \FunctionTok{mean}\NormalTok{(body\_mass\_g, }\AttributeTok{na.rm =}\NormalTok{T),}
            \AttributeTok{desviacion =} \FunctionTok{sd}\NormalTok{(body\_mass\_g, }\AttributeTok{na.rm =}\NormalTok{T),}
            \AttributeTok{simetria =} \FunctionTok{skewness}\NormalTok{(body\_mass\_g, }\AttributeTok{na.rm =}\NormalTok{T),}
            \AttributeTok{curtosis =} \FunctionTok{kurtosis}\NormalTok{(body\_mass\_g, }\AttributeTok{na.rm =}\NormalTok{T))}
\end{Highlighting}
\end{Shaded}

\begin{enumerate}
\def\labelenumi{\arabic{enumi}.}
\setcounter{enumi}{11}
\tightlist
\item
  Usando como guía el libro digital
  \href{https://r-graphics.org/index.html}{R Graphics Cookbook}, genere:
  (i) una distribución de frecuencias con histograma, (ii) una
  distribución de frecuencias basada en densidad, (iii) un diagrama de
  cajas que incluya promedio. En los tres casos la masa corporal debe
  distinguirse por especie.
\end{enumerate}

Recuerde cargar las respuestas a estas preguntas en el formato de Google
Doc generado en el Google Classroom.



\end{document}
